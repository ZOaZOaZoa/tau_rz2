\input{preamble.tex}

\begin{document}
	
	\begin{titlepage}
	\newpage
	\begin{center}
		\includegraphics[width=\textwidth]{tit.png}
		Институт информационных и вычислительных технологий \\
			Кафедра управления и интеллектуальных технологий
		\vspace{1.25cm}
	\end{center}
	
	\vspace{1.2em}
	
	\begin{center}
		%\textsc{\textbf{}}
		\begin{spacing}{1}
			{\Large Расчётное задание \linebreak
			По дисциплине <<Теория автоматического управления>> \\}
			\large{\bf<<Анализ нелинейных систем автоматического управления>>}
		\end{spacing}
	\end{center}
	
	\vspace{5em}
	

	\vspace{6em}
	
		\noindent Выполнил студент: Михайловский М.\,Ю. \\
		Группа: А-03-21 \\
		Вариант: 39\\
		Проверила: Сидорова Е.\,Ю.
	
	
	\vspace{\fill}
	
	\begin{center}
		Москва 2024
	\end{center}
	
\end{titlepage}
	\setcounter{page}{2}
	\pagenumbering{arabic}
	\tableofcontents
	\newpage
	
	\section{Постановка задачи}
	\subsection{Исходные данные}
	
	Дана нелинейная система со структурной схемой представленной на рис.~\ref{scheme}. Нелинейный элемент (НЭ) имеет вид двухпозиционного реле с гистерезисом с параметрами $c = 9,\,B=8$ (рис.~\ref{NE}). Передаточные функции имеют следующий вид:
	\begin{align*}
		&W_1(p) = 0,1; \\
		&W_2(p) = \frac{4}{p^2 + 3p + 9}; \\ 
		&W_3(p) = 3p.
	\end{align*}
	
	\begin{figure}[h]
		\centering\includegraphics[width=.8\textwidth]{схема.png}
		\caption{Структурная схема нелинейной системы}
		\label{scheme}
	\end{figure}
	\begin{figure}[h]
		\centering\includegraphics[width=.4\textwidth]{НЭ.png}
		\caption{Характеристика НЭ вида двухпозиционное реле с гистерезисом}
		\label{NE}
	\end{figure}
	
	\subsection{План исследования}
	
	\begin{enumerate}
		\item Исследовать структуру фазового портрета нелинейной системы. Для этого определить типы фазовых траекторий в различных областях фазовой плоскости. Найти описание границ данных областей, определить координаты равновесных состояний (особых точек) системы. Построить качественно ожидаемый фазовый портрет системы;
		\item С помощью стандартного ППП построить фазовый портрет системы и сравнить его с ожидаемым, полученным в п. 1. Дать заключение о характере возможных процессов в системе и их устойчивости. Определить устойчивость особых точек, наличие автоколебаний. Для трех фазовых траекторий с начальными условиями $(x_{0_1} \neq 0,\;x'_{0_1}=0)$,  $(x_{0_2} = 0,\;x'_{0_2} 
		\neq 0)$ и $(x_{0_3}\neq 0,\;x'_{0_3}\neq 0)$ привести графики изменения процесса $x(t)$ во времени;
		\item Исследовать влияние ширины петли гистерезиса нелинейного элемента на возникновение автоколебаний в системе. Определить значения параметров $c$ и $h$ НЭ 4, при которых имеют место автоколебания (другие параметры системы остаются неизменными, согласно варианту задания; см. примечание ниже). Для этого найти минимальное значение $\lambda_{\min}$ относительной величины ширины петли гистерезиса $\lambda = \dfrac{h-c}{h}\;(h>c)$, при которой возникают автоколебания. Определить амплитуду и
		период автоколебаний при $\lambda = \lambda_{\min}$;
		\item Определить амплитуду и период автоколебаний при увеличении коэффициента
		$K_1$ в 5 раз и значения $\lambda$ в 2 раза относительно $\lambda$;
		\item Провести исследование автоколебаний в системе приближенным амплитудно-частотным методом (методом Гольдфарба). Для этого привести модель системы исходной структурной схемы (рис. \ref{scheme}) к виду модели Гаммерштейна. Построить амплитудно-фазовую характеристику линейной части и инверсную характеристику $[-z(A)]$ эквивалентного
		комплексного коэффициента усиления нелинейного элемента и дать заключение о возможности возникновения автоколебаний в системе данной структуры и их устойчивости. В случае наличия автоколебаний	определить их параметры;
		
		Исследование провести для трёх случаев:
		\begin{itemize}
			\item системы с исходно заданными номером задания параметрами;
			\item системы с параметрами п. 3 при $\lambda = \lambda_{\min}$;
			\item системы с параметрами п. 4.
		\end{itemize}
		\item Сравнить количественно результаты исследования автоколебаний методом фазовой плоскости в п. 2, 3, 4 и методом Гольдфарба в п. 5.
	\end{enumerate}
	
	\section{Исследование методом фазовой плоскости}
	\subsection{Получение фазового портрета системы}
	

\end{document}
